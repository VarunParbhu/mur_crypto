\pagenumbering{arabic}
\chapter{Introduction}
A cryptocurrency is a decentralised digital asset distributed over an extensive network of computers. Bitcoin was the first cryptocurrency to be operational in January 2009 after the appearance of the mysterious paper titled ”Bitcoin: A Peer-to-Peer Electronic Cash System”, published in 2008 under the alias “Satoshi Nakamoto” (a person or group of people) \cite{satoshi2009}. The paper described a peer-to-peer payment system using electronic currency (cryptocurrencies) that could be sent directly from one party to another without using a third party (often a financial institution) to validate the transaction. The main idea behind the paper was to emulate an online shared ledger on the peer-to-peer network to validate all transactions via a blockchain, eliminating the risk of forging the register. Hence, the rise of blockchain technology provides security, privacy and a distributed ledger applicable in IoT, distributed storage systems and many more. In exchange for maintaining the blockchain (run and validate), which is energy dependent on running electronic machines, “miners” are rewarded with cryptocurrency.

\par The value of such a type of currency is highly volatile and speculative. It is different from any other asset on the financial market and thereby creates new possibilities for stakeholders with regard to risk management, portfolio analysis and consumer sentiment analysis \cite{Anne2016}. Bitcoin had a market cap of over \$ 300B in December 2022. From basically nothing in January 2009 to reaching the highest price (known to date) of \$67,566.83 on November 8, 2021, Bitcoin has been highly volatile compared to other traditional forms of FIAT payment. During that time, there have been substantial price changes over short periods. In 2017 the value of a single Bitcoin increased from \$863 on January 9, 2017, to a high of \$17,550 on December 11, 2017 \cite{Abraham2018}. Due to its volatility and never seen behaviour like traditional currencies, Bitcoin (and cryptocurrencies in general) are extremely difficult to predict. Nevertheless, it was found that the best model for Bitcoin price volatility is the AR-CGARCH model, highlighting the significance of including both a short-run and a long-run component of the conditional variance \cite{Paraskevi2017}. Halvor et al. find that the heterogeneous autoregressive model is suitable for Bitcoin volatility whereby trading volume further improves this volatility model \cite{Halvor2019}.

\par In this modern age, social media platforms allow people to share sentiments on a large scale. Hence, prompting research to study the impact of sentiment on different complex problems. Asur and Huberman attempted to solve the revenue prediction of box-office for movies problem using tweet volume and sentiment \cite{Asur2010}. The same kind of data and polling results were also used by Bermingham \& Smeaton to train a linear regression model to predict election results \cite{smeaton2011}. Zhang et al.leveraged sentiments on Twitter to help predict the movement of stock market indices such as Dow Jones, S\&P500 and NASDAQ \cite{Zhang2011}. It was shown that significant negative emotions and opinions caused Dow to go down the next day, and lower negative emotions caused Dow to go up the next day. Consequently, studies are being conducted to correlate the speculative nature of Bitcoin prices and their sentiments on social media. Banerjee et al. studied the impact of cryptocurrency returns and Covid 19-news. A nonlinear technique of transfer entropy was used to investigate the relationship between the top 30 cryptocurrencies by market capitalisation and COVID-19 news sentiment. Results show that COVID-19 news sentiment influences cryptocurrency returns \cite{Banerjee2022}. Zhang et al. explored the cryptocurrency market's reaction to issuers' Twitter sentiments. It was found that cryptocurrency prices react positively to Twitter sentiments \cite{Zhang2022}. In contrast, the trading volume reacts positively to the absolute value of Twitter sentiments in a timely manner (within a period of 24 h).

\par This project introduces the essential concepts of deep learning, optimisation, and sentiment analysis. We then follow up with an application in the prediction of Bitcoin's price. Chapter 2 introduces the perceptron, its development into feedforward neural networks and a derivation of the backpropagation algorithm. In Chapter 3, we looked at the different optimisation algorithms and the application of the Adam algorithm in backpropagation. Then, in Chapter 3, we reviewed the traditional and modern sentiment analysis techniques. An application for the prediction of Bitcoin price using data from different sources, sentiment analysis and LSTM NN is studied in Chapter 5. Finally, the project concludes by summarising all our results, findings, and potential improvements.









%- Discussion around Cryptocurrency and frequency of changes \\
%- Discussion around Twitter and the API (amount of data) \\
%- Discussion around the advancement Compute processing speed and Deep Learning algorithms