\chapter{Sentiment Analysis}
People's opinions, feelings and sentiments towards entities such products, services, other people, events, news, issues, topics, etc. can be in very large volume, complex and difficult to be understood and processed by machines and computers. Thus, sentiment analysis, also known as opinion mining, started to popularised along the rise of social media when large amount of digital text data were suddenly available for mining. Natural language processing (NLP) helps computers process and understand human based language to perform repetitive task. Sentiment analysis is a niche of NLP. It aims at quantifying the positivity, negativity and/or neutrality of implied or expressed in a given text.

Social media have been providing large platforms for people to share their opinion freely and expressed their views on any subject across various geographical and spatial  boundaries. They have also allowed people to connect, influence and be influenced by such opinions and views. These interactions have been studied in the 1940s and 1950s among people in organizations by management science researchers. Since 2002, with social media, those studies have been performed at grand scales with the abundance of data. Thus, advanced sentiment analysis research have been performed in field political science, economics, finance and management science as they are heavily dependent on public opinions. 

Asur and Huberman (2010) (ADD REF) attempted to solve the revenue prediction problem using both the tweet volume and the tweet sentiment. Same kind of data along with polling results were also used, in Bermingham and Smeaton (2011) (ADD REF), to train a linear regression model to predict election results. Zhang et al. (2010) (ADD REF) leveraged sentiments on Twitter to help predict the movement of stock market indices such as Dow Jones, S\&P500 and NASDAQ. It was shown that large negative emotions and opinions caused Dow to go down the next day and lower negative emotions cause Dow to go up on the next day. Bar-Haim et al. (2011) (ADD REF) on one hand leverage sentiments on Twitter but on the other did not treat all Twitter authors equally. Only expert investors were used as features in training stock price movement predictors. Undeniably, modern social media (started early 2000s) have grown into a major influencer of human opinion and sentiments.


%LSTM
%BERT
%RoBERTa